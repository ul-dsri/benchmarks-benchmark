\documentclass{article}

% if you need to pass options to natbib, use, e.g.:
%     \PassOptionsToPackage{numbers, compress}{natbib}
% before loading neurips_2024


% ready for submission
\usepackage{neurips_2024}
\usepackage{tcolorbox}
\usepackage{xcolor}


% to compile a preprint version, e.g., for submission to arXiv, add add the
% [preprint] option:
%     \usepackage[preprint]{neurips_2024}


% to compile a camera-ready version, add the [final] option, e.g.:
%     \usepackage[final]{neurips_2024}


% to avoid loading the natbib package, add option nonatbib:
%    \usepackage[nonatbib]{neurips_2024}


\usepackage[utf8]{inputenc} % allow utf-8 input
\usepackage[T1]{fontenc}    % use 8-bit T1 fonts
\usepackage{hyperref}       % hyperlinks
\usepackage{url}            % simple URL typesetting
\usepackage{booktabs}       % professional-quality tables
\usepackage{amsfonts}       % blackboard math symbols
\usepackage{nicefrac}       % compact symbols for 1/2, etc.
\usepackage{microtype}      % microtypography
\usepackage{xcolor}         % colors


\title{Benchmarks' Benchmark}


% The \author macro works with any number of authors. There are two commands
% used to separate the names and addresses of multiple authors: \And and \AND.
%
% Using \And between authors leaves it to LaTeX to determine where to break the
% lines. Using \AND forces a line break at that point. So, if LaTeX puts 3 of 4
% authors names on the first line, and the last on the second line, try using
% \AND instead of \And before the third author name.


\author{%
  Sean McGregor\thanks{Use footnote for providing further information
    about author (webpage, alternative address)---\emph{not} for acknowledging
    funding agencies.} \\
  UL Research Institutes\\
  \texttt{to-be-determined@example.com} \\
  % examples of more authors
  % \And
  % Coauthor \\
  % Affiliation \\
  % Address \\
  % \texttt{email} \\
  % \AND
  % Coauthor \\
  % Affiliation \\
  % Address \\
  % \texttt{email} \\
  % \And
  % Coauthor \\
  % Affiliation \\
  % Address \\
  % \texttt{email} \\
  % \And
  % Coauthor \\
  % Affiliation \\
  % Address \\
  % \texttt{email} \\
}


\begin{document}


\maketitle


\begin{abstract}
Large language model (LLM) benchmarks enable system use decisions informed by LLM properties, but benchmarks may be rendered unreliable for real world decision making by a variety of threats to benchmark longevity, correctness, coverage, consistency, and intelligibility. Motivated by emerging LLM safety benchmarks, on whose scores people rely on to make decisions impacting real world safety, this work presents a benchmark for LLM benchmarks inspired by National Institute of Standards and Technology risk management processes. High scores indicate a reduced likelihood and/or severity of inappropriate reliance on a benchmark.
\end{abstract}


\section{Executive Summary}

Reliable real world Large Language Model (LLM) benchmarks clearly state their purpose and prevent or mitigate a large number of threats to their reliability. Identifying which benchmarks are reliable is not always possible without insider information on the production and maintenance of the benchmark. The Benchmarks' Benchmark (\(B_{2}\)) provides a means of:

\begin{itemize}
    \item[a.] {\bf identifying which benchmarks are more reliable for real world decisions}
    \item[b.] identifying {\bf open reliability research problems}
    \item[c.] ensuring low cost {\bf non-reliable benchmarks do not dominate} the market
    \item[d.] advancing the development of {\bf rigorous safety standards}
\end{itemize}

Through time, the operational sophistication and \(B_2\) scores of highly reliable benchmark operators are expected to increase. They are currently,

The initial scores are the following,
% table 1

We are pending submission from the following benchmarks,
% table 2

All LLM benchmarks are invited to submit. The absence of a \(B_2\) score for a benchmark indicates the benchmark is produced for research (i.e., not real world reliable) purposes only or the benchmark organization has chosen to not score their benchmarks.

\subsection{Conflicts of Interest}
This research is the product of a large number of independent individuals engaged in benchmarking large language models (LLMs), including contributors to all the benchmarks herein presented. Researchers from the Digital Safety Research Institute (DSRI) of the UL Research Institutes maintain the independence of this work. DSRI's funding arises from Underwriter's Laboratories more than 100 years of running safety testing and certification and has not received external funding for the production of this work or any of the contributing organizations.

\subsection{Disclaimers}
\begin{itemize}
\item {\bf Benchmark scores rely on the representations made by covered benchmarking organizations.}
\item The benchmark benchmark is intended to serve as a guide for producing and adopting best-in-class LLM benchmarks, but this work and its associated scores are not a substitute for learning more about the covered benchmarks and developing an independent sense for their reliability.
\end{itemize}

\section{Introduction}
Benchmarks have played a central role in the rapid development of machine learning systems. New benchmarks are produced, researchers optimize models to the benchmarks, and the capabilities of AI systems advance., While benchmarks as optimization targets have greatly advanced the capacity for ML systems to perform useful tasks, the operational, statistical, and communication requirements for producing benchmarks to influence real world decisions far exceeds the requirements imposed by benchmarks used in optimization. People who inappropriately rely on a benchmark in the real world (e.g., by making a decision about what is a safe use case) will more often harm themselves or others.

Some means of separating those benchmarks intended for research and engineering purposes (i.e., "optimization benchmarks", see definition X) from benchmarks intended to express performance properties in the real world (i.e., "decision benchmarks", see definition X) is required. This work, presented as a benchmark of large language model (LLM) benchmarks named \(B_2\) (i.e., a "benchmarks' benchmark"), applies risk management processes inspired by the National Institute of Standards and Technology (NIST) information security risk management process to score the reliability of LLM benchmarks for real world decision making. In this realization of risk management processes, we identify threats to benchmark reliability (definition X) and invite the benchmark community to express controls and mitigations that are likely to reduce the severity or likelihood of those threats materializing into inappropriate reliance.

{\bf Definition X. Optimization benchmark.} A benchmark that is directly or indirectly the target of optimization for an engineering or research program. This may alternatively be defined as a "research benchmark."

{\bf Definition X. Decision benchmark.} A benchmark produced to inform the decision making of people exploring adoption of an LLM or its application in a particular real world scenario.

\begin{center}
    \begin{tcolorbox}[colback=blue!10, colframe=blue!50, width=\textwidth, boxrule=0.5mm, sharp corners, coltext=black, halign=center]
        \bf "Benchmark Reliability" NOT "Reliability Benchmark"
    \end{tcolorbox}
\end{center}

Threats to benchmark reliability may lead a user to read the benchmark results and make a false inference about the risks and benefits of different LLM system choices.

For safety benchmarks in particular, "human error" as an explanation for inappropriate decision making is not consistent with an engineering ethos that looks to prevent harm by improving systems and processes. In aviation safety, when a pilot fails to safely land a plane. Human error is typically the last of a series of failures in design, maintenance, and other processes leading to a bad decision. Similarly, engineering benchmarks such that the user makes appropriate inferences about benchmarked systems requires careful benchmark preparation and presentation to avoid threats to the reliability properties outlined in Table X.

% table 3
These dimensions are determined by operational, statistical, and communication factors that jointly determine whether a user may rely on a benchmark when making decisions. High scores on \(B_2\) are achieved by means of mitigating or controlling risk and indicate a reduced likelihood and/or severity of inappropriate reliance on a benchmark.

As a relatively new science, LLM benchmarking poses many unquantified or unidentified risks. The longevity of static benchmarks is questionable, and in the case of \(B_2\), new risks will be identified or better understood. Therefore, we treat the introduction of a new risk as an opportunity for benchmarks to further advance the practice of LLM benchmarking.

The following paper introduces the LLM benchmark production process to scope the analysis of LLM benchmark reliability, then details elements of the \(B_2\) production process within the context of risk management processes. The paper concludes with details on the benchmarks that have been assessed to date.

% table 4 - outline?
\section{LLM Benchmark Production}

LLM benchmarks are typically produced in an iterative manner as detailed in Figure X. We briefly introduce these steps in turn below.

% figure 1
% Caption: Figure X. The chain of benchmark and assessment production follows a series of steps identified above. 

{\bf Step 1. Task Definition.} At this stage the benchmarking organization defines the task that the LLM is expected to perform and the desirable (or undesirable) behaviors against which it is being measured. Risk #001, which indicates there is a disconnect between the user's understanding of the benchmark and what the benchmark is actually measuring is a risk to the intelligibility of the benchmark. This categorization implies a critique of many common benchmarks that cover a vast array of capabilities without a definite scope to the benchmark. Such benchmarks greatly advance the capabilities of systems through their generality, but they do not provide a means for forming a mental model of what the benchmark is indicating. Analogously, the user may know an engine's horsepower, but not know whether it is in a car or a boat.

\begin{center}
    \begin{tcolorbox}[colback=gray!10, colframe=black!50, width=\textwidth, boxrule=0.5mm, sharp corners, coltext=black]
        {\bf Example Risk:} \#001
        \newline
        {\bf Description:} "Specified task does not match task performed for user"
    \end{tcolorbox}
\end{center}

{\bf Step 2. Prompt Generation.} Having defined the task the LLM is expected to perform, the next step is to produce data related to that task. Since many LLM developers work with publicly available internet data, one major risk to the correctness of a benchmark is that the benchmark uses publicly available data that is in the training set of the model. Worse yet, data vendors providing prompts consistent with a testing specification may charge the benchmarking organization for data that is publicly available and part of the training program for the benchmarked systems. This particular risk can be mitigated by searching for a select sample of test data within common datasets (e.g., common crawl).

\begin{center}
    \begin{tcolorbox}[colback=gray!10, colframe=black!50, width=\textwidth, boxrule=0.5mm, sharp corners, coltext=black]
        {\bf Example Risk:} \#003
        \newline
        {\bf Description:} "Prompts are collected from publicly available sources and presented as novel"
    \end{tcolorbox}
\end{center}

{\bf Step 3. Prompt Inferencing.} After producing the benchmark dataset, it is time to pipe the data through systems under test (SUTs) to get the outputs. Since many popular LLMs are closely guarded by their companies and only run for users on company-controlled hardware, benchmark prompts are often sent to SUT developers via their public APIs. If the SUT developer then accidentally or intentionally logs the prompts and brings them into their model engineering, the benchmark will no longer be reliable. The risk that the prompts will be exposed to one or more SUT developers is then the sum of the risks expressed across all benchmarked SUTs. Thus a benchmark that only benchmarks SUTs on the benchmark organization's hardware is more reliable, but likely less useful as the most important SUTs to relying persons are likely not to be covered.

\begin{center}
    \begin{tcolorbox}[colback=gray!10, colframe=black!50, width=\textwidth, boxrule=0.5mm, sharp corners, coltext=black]
        {\bf Example Risk:} \#020
        \newline
        {\bf Description:} "Prompts are sent to model vendors when inferencing"
    \end{tcolorbox}
\end{center}

{\bf Step 4. Output Evaluation.} After inference the benchmark dataset, the SUT outputs are typically high dimensional (e.g., full text) and require interpretation consistent with the benchmark's purpose. Often this task is performed by an evaluator model, such as an LLM. Imagine now that an open source LLM is applied for this purpose. Any SUT developer could then place the evaluator LLM into its system chain to achieve a perfect score on the benchmark. Two mitigations are possible for this particular risk. Either the evaluator model can be kept strictly internal to the benchmark evaluation, or the machine evaluator could be replaced entirely by human effort.

\begin{center}
    \begin{tcolorbox}[colback=gray!10, colframe=black!50, width=\textwidth, boxrule=0.5mm, sharp corners, coltext=black]
        {\bf Example Risk:} \#023
        \newline
        {\bf Description:} "SUT developers place evaluator within system chain"
    \end{tcolorbox}
\end{center}

{\bf Step 5. Scoring.} Having labeled each individual SUT output, the next step is to statistically aggregate the responses so they can be presented to the user in some form. One example risk at this step is that important relationships uncovered at the sample level might be hidden in the aggregate. A SUT for English, French, and Hindi might perform well for English and French, while failing spectacularly for Hindi. If the scoring function produces a simple average without propagating the Hindi failure, then the benchmark is not reliable for Hindi use decisions. Such problems can be mitigated by propagating uncertainty, confidence, and exceptions as a data structure to the presentation step.

\begin{center}
    \begin{tcolorbox}[colback=gray!10, colframe=black!50, width=\textwidth, boxrule=0.5mm, sharp corners, coltext=black]
        {\bf Example Risk:} \#029
        \newline
        {\bf Description:} "Failure to propagate uncertainty or confidence from lower level measures to higher level grades"
    \end{tcolorbox}
\end{center}

{\bf Step 6. Grade Presentation.} At the presentation step, the benchmark score is rendered for consumption by the user. The most common presentation at the moment is HuggingFace.co leaderboards, which are typically up to date with the latest LLM releases. Few benchmarks are fully detailed on the HuggingFace platform (e.g., contextualizing each score with information on uncertainty). They also often lack information on what not to rely on the benchmark for. As such, there is a risk the user will not understand the scope of the benchmark as presented and make a false assumption of what an LLM may appropriately be asked to do.

\begin{center}
    \begin{tcolorbox}[colback=gray!10, colframe=black!50, width=\textwidth, boxrule=0.5mm, sharp corners, coltext=black]
        {\bf Example Risk:} \#033
        \newline
        {\bf Description:} "User misunderstands the scope of the benchmark"
    \end{tcolorbox}
\end{center}

{\bf Step 7. Maintenance.} The final step of the benchmark production cycle is maintenance. Benchmark organizations today typically deliver benchmarks and move on to the next problem, but LLMs and the world they act within are constantly changing. So too must the benchmarks. Setting aside the possibility of SUT developers gaming a benchmark to achieve unrealistic scores, users themselves change in their capacities, use cases, and environments. Work establishing the ecological validity of a prompt set for processing current slang will become invalid within months as new words are introduced into prompts.

\begin{center}
    \begin{tcolorbox}[colback=gray!10, colframe=black!50, width=\textwidth, boxrule=0.5mm, sharp corners, coltext=black]
        {\bf Example Risk:} \#036
        \newline
        {\bf Description:} "User behavior shifts through time"
    \end{tcolorbox}
\end{center}

For a user to rely upon benchmark scores, threats to the integrity of the assessment must be addressed through all stages of the benchmark's assembly.

\section{\(B_2\) Design, Interpretation, and Reliability}
\(B_2\) is inspired by the NIST Guide for Conducting Risk Assessments for Information Security.

\begin{center}
    \begin{tcolorbox}[colback=gray!10, colframe=black!50, width=\textwidth, boxrule=0.5mm, sharp corners, coltext=black]
"Risk assessment is one of the fundamental components of an organizational risk management process as described in NIST Special Publication 800-39. Risk assessments are used to identify, estimate, and prioritize risk to organizational operations (i.e., mission, functions, image, and reputation), organizational assets, individuals, other organizations, and the Nation, resulting from the operation and use of information systems. The purpose of risk assessments is to inform decision makers and support risk responses by identifying: (i) relevant threats to organizations or threats directed through organizations against other organizations; (ii) vulnerabilities both internal and external to organizations;(iii) impact (i.e., harm) to organizations that may occur given the potential for threats exploiting vulnerabilities; and (iv) likelihood that harm will occur. The end result is a determination of risk (i.e., typically a function of the degree of harm and likelihood of harm occurring)."
    \end{tcolorbox}
\end{center}

Per NIST guidance, preparing for a risk assessment involves the following steps, which are provided for each scored benchmark as determined by B2.


\begin{itemize}
    \item {\bf Identify the purpose of the assessment:} to identify and mitigate threats to the reliability of the LLM benchmark
    \item {\bf Identify the scope of the assessment:} all threats likely to lead a person to make a false inference about the properties of an LLM along with their mitigations.
    \item {\bf Identify the assumptions and constraints associated with the assessment:} the benchmark organization will faithfully indicate the properties of their benchmark program.
    \item {\bf Identify the sources of information to be used as inputs to the assessment:} insider knowledge about the operations of the benchmarking organization and the properties of the benchmark.
\end{itemize}

The final step is to, {\bf Identify the risk model and analytic approaches (i.e., assessment and analysis approaches) to be employed during the assessment,} which cannot be so compactly stated. Table X presents the collection of terms upon which B2 develops its analytic frame.

% table X

The complete set of risks and mitigations are available in Appendix A, but to present how they are rolled up into scores we will demonstrate for a single threat.

% boxed thing not marked as ready yet

% table X
% caption: Table X. Candidate responses to threat #002, "Prompt writers produce prompts with LLMs."

Now across all threats and their mitigations, this results in the following scores for the currently assessed benchmarks.

% table X
% caption: Table X. Absolute reliability of benchmarks

LLM benchmarking is an evolving science with threats to reliability being identified, characterized, and mitigated through time. As such, B2 is designed to track progress in developing highly-reliable benchmarks, whose adopted mitigations against threats to reliability provide the means for benchmark developers to increase their scores. As unknown risks are identified along with mitigations, the maximum score increases along with the sophistication of LLM benchmarking as a reliable practice. The current list of threats to reliability and their candidate responses are given in Appendix A.

\section{Assessed Benchmarks}
The following benchmarks have been benchmarked.

{\bf ML Commons 1.0}
\begin{itemize}
\item Brief intro to 1.0
\item Strong performing points
\item Weak performing points
\item Open Research Questions (areas requiring additional analysis)
\end{itemize}
{\bf ML Commons 0.5}
\begin{itemize}
\item Brief intro to 0.5
\item Strong performing points
\item Weak performing points
\item Open Research Questions (areas requiring additional analysis)
\end{itemize}

% -----------------------------------------------------------------------------------------------

\end{document}
